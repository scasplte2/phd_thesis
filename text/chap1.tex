\chapter{Introduction}
\label{ch:intro}

The ability to engineer and manipulate quantum states lies at the heart of modern atomic physics experiments using ultracold gases \cite{Davis1995,Anderson1995,Bradley1995,DeMarco1999,Lang2008,Ni2008}. 
Two important tools for this pursuit are Feshbach resonances \cite{Chin2010,Kohler2006} and optical lattices \cite{Bloch2008}. 
This proposal will detail our recent work building and characterizing a three-dimension optical lattice for use with ultracold and quantum degenerate gases of neutral strontium.
Furthermore, we will present the first experiment we hope to pursue with the optical lattice; the creation of Feshbach molecules using an optical Feshbach resonance. 
We will also briefly discuss other future plans such as the production of highly excited ground state Sr$_2$ dimers through adiabatic internal state transfer.

Should probably mention somewhere that this is long-range PA, in contrast to short-range stuff being explored now.

Julienne form of the corss section 6.16 in CM. Discuss how important the phase space density is, the timescale for interactions, and the complex 

PA favors physicist molecules

PA can come in many forms (in a lattice, in a bulk gas, via dissociating molecules) Experimentally we observe PA by looking for trap loss \hl{doublon paper}.
There are multiple flavors of PAS. Can do one-photon or two-photon.
study of simplest molecules

\cite{Aman2018}

\section{Few-body physics}
\label{sec:few-body}

field of photoassociation in ultracold gases, wherein studies of molecular structure have revealed the most accurate descriptions of atomic interactions and have become a fundamental probe of the ultracold toolbox \cite{Jones2006}.



%history
Pioneering work done in the early 90's used PA to interrogate the sturucture of interatomic potentials to deduce the scattering lengths between atoms.

a photoassociation experiment can be used to map the square of the scattering wave function in the ground electronic state at the Condon points corresponding to the different excited bound levels. \cite{Borkowski2009}


 
% recent experiments
rabi oscilations between atomic and molecular condensates (cite ours and the lattice experiment that followed)

short-range PA
This work is focused on long-range PA but in recent years groups have also developed short-range PA techniques for the creation of rovibrational ground state molecules. These techniques rely heavily on favorable overlap integrals betwwen molecular wavefunctions and typically searching for favorable intermediate states is a pain (that is why our large FCF might be useful)


Most of what we know about quantum mechanics comes from either scattering experiments or spectroscopy.

\section{Halo molecules}
\label{sec:halo}

Mostly studied in helium

Comes from bound state of the dirac potential. Unsure how much detail I want right here

Check out CM Juleinne pg 229. He has a ref

Understanding the potential is hard
	weakly bound dimers probe the long range part of potential
	halo dimers especially are 
	
	
\begin{figure}
\label{fig:1boundstate}
	\centerline{
	\includegraphics[width=\textwidth]{toy_LJ_condon.png}}
	\caption{Example bound states}{}
\end{figure} 

\begin{figure}
\label{fig:1halo}
	\centerline{
	\includegraphics[width=\textwidth]{toyLJ_halo.png}}
	\caption{Example halo state}{}
\end{figure} 

%% paper
Weakly bound ground-state dimers are of great interest in ultracold atomic and molecular physics. 

In the extreme case of a scattering resonance, the least-bound state represents an example of a quantum halo system \cite{jrf04} with spatial extent well into the classically forbidden region. 

Halo molecules show universality, meaning that molecular properties such as size and binding energy can be parameterized by a single quantity, the $s$-wave scattering length $a$, independent of other details of the atom-pair interaction \cite{kgj06,bha06}. 

For potentials that asymptote to a van-der-Waals form, an additional parameter, the van der Waals length $l_{\mathrm{vdW}}$, can be introduced for a more accurate description. 

Efimov trimers also exist in systems near a scattering resonance, influencing dimer and atomic scattering properties and introducing additional universal phenomena \cite{bha07,nen17}. 

Ultracold halo molecules are often associated with magnetic Feshbach resonances \cite{cgj10}, for which the scattering state and a bound molecular state can be brought near resonance by tuning a magnetic field.

This is a naturally occurring halo molecule, meaning it exists in the absence of tuning with a magnetic Feshbach resonance. 
A well-known example of a naturally occurring halo molecule is the $^4$He$_2$ dimer \cite{lmk93,sto94,kgj06}.
Moreover, the least-bound vibrational level of the ground state of $^{40}$Ca$_2$, which was recently studied using similar methods \cite{Pachomow2017a}, is similarly related to this regime.

There are important differences between halo molecules associated with magnetic Feshbach resonances and the naturally occurring halo molecule in $^{86}$Sr. 
With magnetic Feshbach resonances, the relevant scattering and bound molecular states lie on different molecular potentials, and single-photon magnetic-dipole transitions can be used to measure molecular binding energies with RF or microwave spectroscopy \cite{cgj10,cju05,thw05b}. 
Typically, this is done by first forming molecules through magneto-association and then driving bound-free or bound-bound transitions converting the halo molecule into a different state.
Other methods include spectroscopy with an oscillating magnetic field \cite{thw05b}, a modulated optically controlled Feshbach resonance \cite{chx15}, and Ramsey-type measurements of atom-molecule oscillation frequencies \cite{ckt03}. 
It is also possible to efficiently populate halo states with a magnetic-field sweep \cite{grj03} or evaporative cooling \cite{jba03} near a magnetic Feshbach resonance \cite{cgj10}. 
These are powerful techniques for manipulating quantum gases of alkali metals and other open-shell atoms, for which there are many magnetic Feshbach resonances. 
Strontium, however, due to its closed-shell electronic structure, lacks magnetic Feshbach resonances in the electronic ground state.



\section{Properties of strontium}
\label{sec:sr}

Therefore, photoassociation relative to the narrow 1S0 to 3P1 transition in Sr can be performed with precisions on the order of kHz. Previous narrow line photoassociation spectroscopy (PAS) has been performed in 88Sr [40, 43], 86Sr [177], and in 84Sr [50]. In addition, two-color photoassociation of the 1S0 to 3P1 line in 88Sr was used to measure the scattering lengths of all the strontium isotopes [42] and several subradiant 1g states have been probed in 88Sr [178]. The ground [96, 179] and excited [180] state molecular potentials have also been explored by Fourier transform spectroscopy.
In addition to probing the shapes ofthe molecular potentials, photoassociation efforts
are motivated by interest in creating ground-state molecules. Ground state molecules have been proposed as a platform for precision measurements, for example to study deviations of the proton-electron mass ratio [181, 182], and/or the fine structure constant [183]. The production of ground state molecules has been demonstrated by decay from excited- molecular states in 88Sr [51] and by using stimulated Raman adiabatic passage (STIRAP) in 84Sr [50,52,184]. In particular, the technique in [184] may offer a path towards creating a molecular BEC.
173, 180



	

The experiments in this proposal will be realized using an ultracold gas of atomic strontium. Fig.\;\ref{fig:energy_level_diagram} shows all of the stable isotopes of strontium, their natural abundance, as well as their inter-particle scattering lengths. The isotopic differences in strontium have important implications for their use in certain experiments. For example, none of the bosonic isotopes of strontium ($^{88}$Sr, $^{86}$Sr, or $^{84}$Sr) display hyperfine structure since they have no nuclear spin, $\vec{I}=0$. However, the fermionic isotope $^{87}$Sr has a large nuclear spin, $\vec{I}=9/2$, which makes it an ideal candidate for exploring exotic phases of quantum magnetism \cite{Beverland2016,Cazalilla2014,Chen2015}. In the studies presented in this proposal, we are sensitive to the isotopic shifts of the bosonic photoassociation lines along the $^1S_0\!\rightarrow\!^3P_1$ transition as well as the various interspecies scattering lengths.

	\begin{figure} 
		\centerline{
		\includegraphics[height=0.4\textheight]{energy_level_diagram.png}}
		\caption{Partial energy level diagram of strontium}{Shown are the relevant transitions and decay rates utilized to perform laser cooling and spectroscopy.}
		\label{fig:energyLevels}
	\end{figure}



\section{Thesis Outline}
\label{sec:outline}

