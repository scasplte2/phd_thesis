\chapter{Photoassociation spectroscopy: theory and setup}
\label{ch:chap3}

\section{Introduction}
\label{sec:pas_intro}

\subsection{Low energy two-body scattering}
\label{ssec:scattering}

consider the two particle system as a single entagled particle
	long range part of this quasi-particle is just the eigenstates of the separate particles themselves (only composed of two parts)
	but the short range part is going to be determined by some complex physics (new eigenstates, what is the coupling mechanism?)
		the vdW point is the boundary distance?
		coupling is due to the interatomic potential, there is at least the long-range part falling off as R6, what are the types of interactions which make up the internal wall?


Think I want to introduce the photoassocation by talking about the collisional wavefunction

what will that do?

I want to build up ideas about the FCF and need the wf for that
	to get qf I have to go back to scattering theory

ideas of the wavefunction become that basis for how you want to talk about interacting potentials


free atoms
scatering as single particle state (differnet eigenstate)
	interaction determined by some gnarly stuff
From scattering theory we know that the long range behavior is determined by short range physics
	how do we know this? (the dalibard intro)
Can we come up with good enough pseudo potentials to describe the short range physics and then solve the schrodinger equation to extract wavefunctions?
	we want wavefunctions because that is the full characterization
	we don't know the right eigenbasis for the short range part but we can make some guesses (in particular Hund cases setup eigen states for various possible internal states)
	Bohn and Julienne theory guessed based on using quantum defect theory
		this pre-supposes that the bound and free wavefunctions are similar (I forget in what respect) but that the bound ones must go to zero as R->Inf
If we have some notion of the wf then we can construct matricies which define interactions once we add additional coupling to the scattering problem


now in a position where I need to connect scattering theory and the PAS


Once we have the ground state wavefunction of our new particle we can construct the internal structure by considering the internal energy strucutrue of the constituent atoms
	Can I make a connnection that since it is a composite particle we must consider all the various configurations available?
	
can I save?

\subsection{Modifying interactions}
\label{ssec:mod_int}

\subsection{PAS in atomic physics}
\label{ssec:pas_amo}

\section{Semi-classical treatment of lineshapes}
\label{sec:bohn_and_julienne}

\section{Observing photoassociative loss}
\label{sec:pa_methods}

This is a test file for chapter 3

ok, I need to think how I am going to introduce photoassociation theory? It has to be qquick. I am never going to get this done. 

Okk, first ogg we introduce the potential between atoms

this arises from the interaction of scatteri ng theory with a molecular state. ''the interaction potential between two atoms. Ehich is caused by?

this results in a poitential that supports bound states . In atomic physics, our low density fases are mainly within the regime of small interactions. This spatial dependence is mapped onto the internal energy levels of each atom. I want to say dressed state model here (review atom-photon coupling, atomic physics book).  