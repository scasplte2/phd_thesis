\chapter{Conclusion}
\label{ch:conclusion}

%% conclusion from halo paper
We have measured the binding energy of the least-bound vibrational level of the ground electronic state of the $^{86}$Sr$_2$ molecule with two-photon photoassociative spectroscopy. Using the universal prediction for the binding energy of a halo state including corrections derived for a van der Waals potential [Eq.\ (\ref{Eq:BindingEnergyGao})] \cite{gfl93,gao01,gao04}, we extract an improved value of the $s$-wave scattering length.

We also characterized the AC Stark shift of the halo-state binding energy due to light near resonant with the single-photon photoassociation transition. A model only accounting for a single excited-state channel \cite{bju96} cannot explain the observed frequency dependence of the AC Stark shift, which can be attributed to the proximity of other excited states.

Large AC Stark shifts of the halo state point to the possibility of optically tuning the $^{86}$Sr scattering length, similar to recent demonstrations of optical tuning of magnetic Feshbach resonances \cite{blv09,chx15}. This is attractive because ground-state strontium lacks magnetic Feshbach resonances. With improved measurement of the photoassociation resonance frequency and its dependence on background atom density, perhaps combined with optical manipulation of the scattering length, it may also be possible to study the landscape of Efimov trimers associated with this naturally occurring scattering resonance. This work also points to the need for improved theory, such as an improved calculation of the Sr ground-state molecular potential and $C_6$ coefficient, which could be compared with this high-accuracy measurement of the halo binding energy.

%internal state transfer combined with optical Feshbach resonance.

%strong transition and large AC Stark shift suggests optical manipulation of the scattering resonance might be promising in this system.

%Efficient optical production of weakly bound Sr$_2$ molecules \cite{cbc17} %Schreck paper
%is also of interest for metrology \cite{zky08}.


The work presented in this proposal is a natural extension of previous work done in our lab using an optical Feshbach resonance and one color photoassociation to manipulate the quantum state of a Bose-Einstein condensate. The creation and characterization of a novel type of Feshbach molecule is of fundamental interest to complete the analogy between optical and magnetic Feshbach resonances as well as to test the mechanism of Feshbach molecule stability in the presence of closed channel decay. This experiment provides a practical first demonstration of an optical lattice on our apparatus, which can be readily extended to a number of experiments such as out of equilibrium unitarity quenches \cite{Makotyn2014}, strongly interacting Bose gases stabilized by the quantum zeno effect \cite{Fischer2001,Zhu2014,Daley2009,Syassen2008}, and the observation of exotic spin phases \cite{Beverland2016,Cazalilla2014,Chen2015}. Moreover, additional insight might also be drawn from revisiting OFR and photoassociation in an optical lattice and employing new measurement techniques in the lattice \cite{Taie2016}. These and future experiments will take advantage of the variety of interactions and narrow intercombination transitions available in strontium as well as the control and selectivity afforded through an optical lattice.