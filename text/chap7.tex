\chapter{Conclusion} \label{ch:conclusion}
This thesis presents studies probing the least-bound vibrational level of the ground electronic state of the $^{86}$Sr$_2$ molecule using two-photon photoassociative spectroscopy.
The large $s$-wave scattering length of $^{86}$Sr reveals this to be a naturally occurring halo molecular state.
Using a precise measurement of the binding energy of this extremely weakly-bound state, we estimate an improved value of the strontium long-range $C_6$ coefficient from the best-known descriptions of the ground state X$^1\Sigma_g^+$ potential and the universal prediction for the binding energy of a halo state, accounting for corrections derived for a van der Waals potential.
This work points to the need for an improved theoretical understanding of the Sr ground state molecular potential, which could be compared with this high-accuracy measurement of the halo binding energy.
Despite uncertainty in the applicable model potential, we use mass-scaling to predict the $s$-wave scattering lengths for each isotopic combination of strontium and find them to be model invariant for the chosen description of the strontium ground state potential.

We also characterized the AC Stark shift of the halo state binding energy due to light near-resonant with the single-photon photoassociation transition. 
Through numerical simulation of a three-level model and focusing on a regime where the AC Stark shift is comparable to the halo binding energy, we find it is important to consider intermediate state coupling due to both excitation lasers.
Subsequent analysis of this theoretical model using Floquet theory and a perturbative expansion about the halo resonance position allowed us to derive a simple analytic form for the binding energy which showed that the light shift is a sum of independent AC Stark shifts from each of the excitation lasers \cite{Wynar2000,Borkowski2009,Tojo2006}.
These simulations also predict higher-order loss structures arising from multi-photon resonant processes, as was also observed in our experiments.
However, a three-level model only accounting for a single excited state channel \cite{Bohn1996} cannot explain the full range of the observed frequency dependence of the AC Stark shift, which may be attributed to the proximity of nearby excited states.

Large AC Stark shifts of the halo state point to the possibility of optically tuning the $^{86}$Sr scattering length, similar to recent demonstrations of optical tuning of magnetic Feshbach resonances \cite{blv09,chx15}. 
This is attractive because ground state strontium lacks magnetic Feshbach resonances.
With improved measurement of the photoassociation resonance frequency and its dependence on background atom density, perhaps combined with optical manipulation of the scattering length, it may be possible to study the landscape of Efimov trimers associated with this naturally occurring scattering resonance \cite{nen17,mwc17,bha07,wie12}.
Experiments such as these are natural extensions of previous work from our lab controlling states of quantum matter using an optical Feshbach resonance and coherent one-color photoassociation \cite{Hofer2015,Nicholson2015a,Yan2013c, Yan2013b}.

Although not as well-studied as other strontium isotopes, $^{86}$Sr is a promising candidate for the creation of a novel type of Feshbach molecule originating from the manipulation of the halo state energy via optical coupling.
This is of fundamental interest to complete the analogy between optical and magnetic Feshbach resonances.
Studies in this regime include measuring the binding energy of these optical Feshbach molecules by modulating the laser frequency at a fixed detuning as was done with MFRs \cite{Kohler2006,smg06,Kohler2005}. 
This will allow us to test the binding energy of Feshbach molecules near the universal regime as well as to explore scaling with coupling strength through variation of the laser intensity \cite{Chin2010,Jones2006,Nicholson2015a,Pachomov2017,Reschovsky}.

Related to the creation of Feshbach molecules in the universal regime, adiabatic internal state transfer presents an intriguing possibility of creating large populations of $^{86}$Sr$_2$ halo molecules \cite{Quemener2017,Quemener2012,Lang2008}.
These dimers are interesting intermediaries for further study using STIRAP or related protocols to create deeply bound or absolute ro-vibrational ground state strontium molecules, which can serve as excellent metrological tools \cite{dvm04, Skomorowski2012, zky08, Stellmer2012, Reinaudi2012}.

Moreover, additional insight might also be drawn from revisiting OFR and photoassociation in an optical lattice and employing new measurement techniques in the lattice \cite{Taie2016}. 
By working in an optical lattice with fewer than three atoms per site, we may eliminate atom-molecule and molecule-molecule collisions and accurately measure molecular lifetimes \cite{Thalhammer2006,lsb08} allowing us to test our understanding of the mechanisms that provide stability to Feshbach molecules when the closed channel is naturally decaying \cite{Kohler2005}.
Halo molecule formation in sites with two atoms can be nearly 100\% efficient with MFRs \cite{Mies2000,Chin2010}, and we expect OFRs should behave similarly.
Using the fast temporal control of optical fields and the spatial confinement provided by a lattice we may also explore out of equilibrium unitarity quenches \cite{Makotyn2014} and strongly interacting Bose gases stabilized by the quantum Zeno effect \cite{fgr01,Zhu2014,Daley2009,Syassen2008}.

Finally, utilizing recently developed tools for photoassociation in our lab, we have begun probing the photoassociation spectrum of $^{87}$Sr.
We have observed several molecular transitions targeting the $^1S_0\,\rightarrow\,^1P_1$ transition in a regime of large detuning from the asymptotic transition frequency, $\approx 4-12\,$GHz, where the effects of the small hyperfine splitting of the $^1P_1$, $\approx \pm30\,$MHz, are not resolved.
With further investigation of this spectrum, we hope to develop improved predictions of the positions of molecular states near the $^1S_0\,(F=9/2)\,\rightarrow\,^3P_1\,(F'=9/2)$ intercombination transition which has not yet been successfully demonstrated \cite{Reschovsky}.
A recent study of intercombination line photoassociation of $^{173}$Yb suggests that the $^{87}$Sr intercombination line PAS spectrum is likely to exhibit rich structure \cite{Han2018a, Franchi2017,Kim2016}.
Photoassociation of degenerate or nearly degenerate gases of fermionic strontium-87 loaded into an optical lattice may be the key to overcoming the low excitation strength of the bound-state transitions \cite{bhm12,tys12}.
It is anticipated that these resonances are likely to be individually resolved \cite{Han2018a} and therefore may prove invaluable in studies of quantum magnetism as a means of breaking the naturally occurring $SU(N)$ symmetry.
Even further still, with the installation of an additional one-dimensional lattice at twice the wavelength of our current lattice, $2\times 532\,\text{nm} = 1064\,$nm, we hope to detect exotic spin phases unique to these highly degenerate systems \cite{Beverland2016,cre14,Chen2015, pbl08, Gorshkov2010,Scazza2014, ftc07,Cappellini2014}. 

%This will be achieved by controlling the lattice laser phase to adiabatically combine plaquettes of highly degenerate interacting fermions initially occupying the lowest Bloch band.
%This technique for dynamically changing the lattice topology has been proven to be a remarkably robust experimental procedure \cite{wse13, saj06, alb07, woh10, guj13}.
%This process was previously used to observe the effects of spin-exchange interaction between two spatially separated BECs \hl{porto?}. 

The experiments presented in this thesis and our planned future work demonstrate the versatility and variety of interactions accessible in ultracold and quantum degenerate neutral strontium.
When loaded into optical latices and illuminated with near-resonant laser light, an exceptional degree of dynamic spatial control and state selectivity can be achieved, particularly when coupling through narrow intercombination transitions.
This ability to engineer and manipulate novel quantum states for investigating the extremes of universality and collisions of particles with large symmetries, makes strontium an invaluable experimental probe for exploring questions of fundamental physics concerning the interactions of few-body systems.



%\hl{ref, universality and SU(N) mapping to nuclear}
%



%Powerful tools,
%5Utilizing tools, such as optical lattices, provide an unparalleled degree of dynamic spatial control and state selectivity when combined with near-resonant light,
%The engineering and manipulation of novel quantum states to explore the
%
%
%something about PA
%	engineer and manipulatee quantum states through internal coupling
%something about lattices
%	dynamic spatial control and selectivity
%	
%narrow intercombination transitions available in strontium.
%%%% 87 stuff
%for studies of the photoassociation spectra of the fermionic 87Sr isotope, which has not been successfully investigated at this time. 
%As illustrated by a similar study with 173Yb [196], the spectrum is likely to be very complicated so a good theoretical understanding of the potentials will greatly aid the analysis. 
%Despite the challenges, studying this spectrum should lead to important insights into the collisional properties of fermionic strontium. 
%Since resonances corresponding to different total angular momentum are likely to be individually resolved [196], 
%it may be possible to engineer spin-state dependent optical Feshbach resonances. 
%Such a tool may be useful in a quantum simulation schemes as a method of controllably breaking the SU(N) symmetry of the system.
%and the observation of exotic spin phases . 
%These and future experiments will take advantage of the variety of interactions and narrow intercombination transitions available in strontium as well as the control and s
%
%
%electivity afforded through an optical lattice.
%
%
% using $^{86}$Sr which exhibits a weakly-bound molecular state in the $^1S_0\,+\,^1S_0$ potential with a binding energy of $\approx 65\,$kHz. 
%This vibrational state should be easily accessible through an adiabatic field sweep resulting in highly excited ground state molecules. 
%
%
%
%By adiabatically ramping the laser detuning in an OFR with this isotope, we should be able to transfer signicant population to this state. 
%By varying the speed at which we ramp through the second avoided crossing we should be able to traverse it diabatically or adiabatically [72] and characterize the bound-bound laser coupling.
%The stable ground state would be an ideal intermediate for subsequent de-excitation to more deeply bound levels through optical Raman transitions [69, 70, 71, 28], which may be interesting for exploring ultracold chemistry [1, 6] or for using precision molecular spectroscopy to look for variations of fundamental constants [74].
%
%
%
%But this proposal focuses on the use of OFRs to create, study, and manipulate halo molecules formed by laser coupling of open and closed channels. 
%At a basic level, we wish to produce a new form of halo molecule in which the laser is crucial to the binding and determine if its properties resemble the remarkable properties of halo molecules created with MFRs. 
%
%The coherent control of products in chemical reactions and collisions is a broad and important eld [68].
%In the ultracold context, motional degrees of freedom, in addition to all internal degrees, can be controlled because collisions are in a single partial wave.
%This is related to the scientic goal of producing quantum degenerate samples of deeply bound, ground-electronicstate molecules [1, 5, 28, 6].
%Several methods have been developed for this form of coherent control, invovling optical Raman transitions [69, 70, 71, 28], RF or microwave elds [62], and magnetic eld ramps [72, 62].
%The adiabatic creation of weakly-bound halo molecules from free atoms by slowly sweeping the magnetic eld in MFRs [21, 22, 23, 24, 25, 26, 4] is a crucial rst step for all of these techniques.
%Subsequent eld ramps through avoided crossings between closed-channel and more deeply bound ground molecular states provide incredible control to coherently walk through" the state manifold and produce a desired nal product state [72, 26]. (Fig. 3 shows a simple example).
%We will explore the possibility of applying this technique to produce stable groundelectronic- state molecules by ramping the detuning in an OFR.
%Figure 3 shows the nearthreshold energy levels for an OFR in 86Sr. 
%This isotope has a very weakly-bound molecular level on its ground electronic potential (Eb=h = 65 kHz), which is a naturally occurring halo dimer state that is responsible for the large atom-atom scattering length a = 800 a0
%[73]. 
%
%%% ground state molecules
%Efficient optical production of weakly-bound Sr$_2$ molecules \hl{\cite{cbc17}} %Schreck paper
%is also of interest for metrology \hl{\cite{zky08}}.
%
%These techniques rely heavily on favorable overlap integrals betwwen molecular wavefunctions and typically searching for favorable intermediate states is a pain (that is why our large FCF might be %useful)