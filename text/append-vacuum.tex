\chapter{Neutral apparatus }
"Lorem ipsum dolor sit amet, consectetur adipiscing elit, sed do eiusmod tempor incididunt ut labore et dolore magna aliqua. Ut enim ad minim veniam, quis nostrud exercitation ullamco laboris nisi ut aliquip ex ea commodo consequat. Duis aute irure dolor in reprehenderit in voluptate velit esse cillum dolore eu fugiat nulla pariatur. Excepteur sint occaecat cupidatat non proident, sunt in culpa qui officia deserunt mollit anim id est laborum."

\section{Opening vacuum - data and process}

\section{Nozzle redesign - neuNozzle 2018}

%%%%%%%%
%Z:\Neutral\Laboratory Systems\Vacuum Chamber\2017 - Nozzle Redesign - NUI nozzle

Data about firerod
I had some trouble finding the part number associated with the firerod in the Neutral chamber but luckily I found one of the broken firerods still had it's part number on it, SK7J-2953. 
Below is the info from Valin Corporation who seems to be the local reseller of Watlow products.

SPECIAL DIAMTER
HT FIREROD   
T/C CENTER CORE LOC “A” TYPE “J”
 
120 Volts
240 watts
0.580 +/- 0.004 Diameter firerod
7.5” length
12” of MGT leads
12” of TC leads
6 13/32” of no heat section at lead end.
Crimped of leads construction.

Don't forget the plan was to incorporate a design feature from Plasma's nozzle redesign which addressed the fragility of the feedthrough connection to the heater wire. This was a problem because the heater wire connection is very thin and we used large clamp type connection for them before which was problematic due to the fragility of the heater wire and the necessity that the bulky clamp couldn't touch the nozzle body (as this would short the heater connection). The new connection would allow us to use a smaller crimp to the heater wire, use the rigidity of the feedthrough itself, and use alumina screws to insulate the connection from the nozzle body.

Other reason for redesign was to attach the heat shield directly to the flange.

All the pieces were ordered and built but a mistake on turning the nozzle body meant that we couldn't use the nozzle we built. As of April 2019, this construction is located in \hl{somewhere}