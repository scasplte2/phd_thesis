\chapter{Two-particle momentum probability distribution}
\label{ch:momDistDer}

\section{Standard form}
\label{sec:standardDist}

Typical derivation of relative momentum probability distribution function

We begin by considering the single particle momentum probability distritbuion function (gotten how)? 
Single particle momentum probability distribution
\begin{equation} 
\label{eq:single_particle_prob}
		 f^1( \vec{p} ) = \left(\frac{1}{2 \pi k_B T}\right)^{3/2} e^{\left(\frac{-p^2}{2 m k_B T}\right)}
\end{equation}

\noindent
Extension of this simple Boltzmann equation into the two-particle regime is complicated due to dependence of each particle on the others. If however, we make the assumption that particle collisions are rapid (on some timescale) we can approximate the two particle momentum distribtuion as the product of two single particle functions. This is known as the moleculear chaos assumption and is important for \hl{what???}

The two particle distribution for a homegeneous system is then
\begin{equation}
\label{eq:two_particle_prob}
\begin{split}
		 f^2( \vec{p}_1, \vec{p}_2 ) &= f^1( \vec{p}_1 ) f^1( \vec{p}_2 ) \\
		  &= \left(\frac{1}{2 \pi m k_B T}\right)^3 \exp{\frac{-(p_1^2 + p_2^2)}{2 m k_B T}}
\end{split}
\end{equation}

Next, we'd like to consider a center-of-mass frame for the distribution. So we'll define

 we define the relative and center-of-mass momenta of the two particles by defining

\begin{align*}
	\vec{P}_c & = \vec{p}_1 + \vec{p}_2             &	M   &= m_1 + m_2 = 2m \\
	\vec{p}_r & = \frac{\vec{p}_1 - \vec{p}_2}{2}   &   \mu &= \frac{m_1 m_2}{m_1 + m_2} = \frac{m}{2}
\end{align*}

from these equations we can use conservation of energy to determine the quadrature sum of the two momenta

\begin{align*}
	\frac{p_1^2}{2m} + \frac{p_2^2}{2m} &= \frac{P_c^2}{2M} + \frac{p_r^2}{2\mu} \\
	p_1^2 + p_2^2 &= \frac{P_c^2}{2} + 2 p_r^2
\end{align*}

thus the momentum probability distribution take the form

\begin{equation}
\label{eq:two_particle_prob_inf_atomFrame}
		 f^2( \vec{P}_c, \vec{p}_r ) = \left(\frac{1}{2 \pi M k_B T}\right)^{3/2} \left(\frac{1}{2 \pi \mu k_B T}\right)^{3/2} 
		 \exp{\frac{-P_c^2}{2 M k_B T}} \exp{\frac{-p_r^2}{2 \mu k_B T}}
\end{equation}

\section{Truncated form}
\label{sec:truncDist}

Here I will derive, motivate, and test limiting cases. Plots showing the effects of truncation will be in the main text

Two particle distribution (for correcting notation, use C and R when denoting CoM and Rel)
\begin{equation}
\begin{split}
	f^2_{\vec{r},trunc} (\vec{p}_1, \vec{p}_2) &= A^2 \left(\frac{1}{2 \pi m k_B T}\right)^3 \exp{ \frac{-(p_1^2 + p_2^2)}{2 m k_B T}} \\ &\quad \times \heaviside{\epsilon_{max} - U(\vec{r}) - \frac{p_1^2}{2m}} \heaviside{\epsilon_{max} - U(\vec{r}) - \frac{p_2^2}{2m}}
\end{split}
\end{equation}

We have introduced a normalization constant $A$ here to ensure the that integration over the truncated probability distribution remains equal to one.

The meaning of $_r$ is such that f should be evaluated at each point in space. Furthermore since the atoms are held in a trapping potential. each point in space has a local trap depth relative to the lip at the top of the trap \hl{need some figure to try and denote this}

Want distribution of relative momenta so integrate out center of mass. Going to drop the two and trunc for now
\begin{equation}
\begin{split}
	 \tilde{f}_{\vec{r}}(\vec{p}_{rel}) & = \int d^3 \vec{P}_c \; f_{\vec{r}} (\vec{p}_1, \vec{p}_2) \\
	 &= \left(\frac{1}{2 \pi M k_B T}\right)^{3/2} \left(\frac{1}{2 \pi \mu k_B T}\right)^{3/2} A^2 \int d^3 \vec{P}_c \; e^{\left(\frac{-P_c^2}{2 M k_B T}\right)} e^{\left(\frac{-p_r^2}{2 \mu k_B T}\right)} \\ 
	 &\quad \times \; \heaviside{\epsilon_{max} - U(\vec{r}) - \frac{P_c^2}{8m} - \frac{p^2}{2m} - \frac{\vec{P}_c \cdot \vec{p}}{2m}} \heaviside{\epsilon_{max} - U(\vec{r}) - \frac{P_c^2}{8m} - \frac{p^2}{2m} + \frac{\vec{P}_c \cdot \vec{p}}{2m}} 
\end{split}
\end{equation}

Spherically symetrix collisions so can integrate by transforming into spherical coordinates with the radius aligned along the interatomic axis

\begin{equation}
\begin{split}
	 \tilde{f}_{\vec{r}}(\vec{p}) &= \left(\frac{1}{2 \pi M k_B T}\right)^{3/2} \left(\frac{1}{2 \pi \mu k_B T}\right)^{3/2} e^{\left(\frac{-p_r^2}{2 \mu k_B T}\right)} A^2  \int_0^{\pi} \sin \theta d\theta \int_0^{2\pi} d\phi \int_0^{\infty} dP_c \; P_c^2 \; e^{\left(\frac{-P_c^2}{2 M k_B T}\right)} \\ 
	 &\quad \times \; \heaviside{\epsilon_{max} - U(\vec{r}) - \frac{P_c^2}{8m} - \frac{p^2}{2m} - \frac{P_c \, p \cos \theta}{2m}} \heaviside{\epsilon_{max} - U(\vec{r}) - \frac{P_c^2}{8m} - \frac{p^2}{2m} + \frac{P_c \, p \cos \theta}{2m}} 
\end{split}
\end{equation}

\begin{equation*}
\begin{split}
	X  &= \cos \theta \\
	dX &= - \sin \theta d \theta
\end{split}
\end{equation*}

Substitute and integrate over $\phi$

\begin{equation}
\begin{split}
	 \tilde{f}_{\vec{r}}(\vec{p}) &= \left(\frac{1}{2 \pi M k_B T}\right)^{3/2} \left(\frac{1}{2 \pi \mu k_B T}\right)^{3/2} e^{\left(\frac{-p_r^2}{2 \mu k_B T}\right)} 2 \pi A^2  \int_{-1}^1 dX \int_0^{\infty} dP_c \; P_c^2 \; e^{\left(\frac{-P_c^2}{2 M k_B T}\right)} \\ 
	 &\quad \times \; \heaviside{\epsilon_{max} - U(\vec{r}) - \frac{P_c^2}{8m} - \frac{p^2}{2m} - \frac{P_c \, p X}{2m}} \heaviside{\epsilon_{max} - U(\vec{r}) - \frac{P_c^2}{8m} - \frac{p^2}{2m} + \frac{P_c \, p X}{2m}} 
\end{split}
\end{equation}

Recognize that the Heaviside functions cancel each other out on either side of zero, so can eliminate one of the Heavisides and multiply by 2

\begin{equation}
\begin{split}
	 \tilde{f}_{\vec{r}}(\vec{p}) &= \left(\frac{1}{2 \pi M k_B T}\right)^{3/2} \left(\frac{1}{2 \pi \mu k_B T}\right)^{3/2} e^{\left(\frac{-p_r^2}{2 \mu k_B T}\right)} 4 \pi A^2 \int_0^1 dX \int_0^{\infty} dP_c \; P_c^2 \; e^{\left(\frac{-P_c^2}{2 M k_B T}\right)} \\ 
	 &\quad \times \; \heaviside{\epsilon_{max} - U(\vec{r}) - \frac{P_c^2}{8m} - \frac{p^2}{2m} - \frac{P_c \, p X}{2m}} 
\end{split}
\end{equation}

From here we can rewrite using the infinite relative momentum probability distribution $f_{\vec{r}, \infty} (\vec{p})$ from \hl{some equation}

\begin{equation}
\begin{split}
	 \tilde{f}_{\vec{r}}(\vec{p}) &= \left(\frac{1}{2 \pi \mu k_B T}\right)^{3/2} e^{\left(\frac{-p_r^2}{2 \mu k_B T}\right)} \mathcal{G}(T, \epsilon_{max}, p_{rel}) \\
	 &= f_{\vec{r}, \infty} (\vec{p}) \mathcal{G}(T, \epsilon_{max}, p_{rel})
\end{split}
\end{equation}

where $\mathcal{G}(T, \epsilon_{max}, p_{rel})$ is given by

\begin{equation} \label{eq:mom_dist_st_3}
\begin{split}
	\mathcal{G}(T, \epsilon_{max}, p_{rel}) &= A^2 \left(\frac{4 \pi}{2 \pi M k_B T}\right)^{3/2} \int_0^1 dX \int_0^{\infty} dP_c \; P_c^2 \; e^{\left(\frac{-P_c^2}{2 M k_B T}\right)} \\ 
	 &\quad \times \; \heaviside{\epsilon_{max} - U(\vec{r}) - \frac{P_c^2}{8m} - \frac{p^2}{2m} - \frac{P_c \, p X}{2m}} 
\end{split}
\end{equation}

Now define two dimensionless variables $\tilde{\epsilon}$ and $\tilde{E}$ which will be used to change variables once more

\begin{align*}
	\tilde{\epsilon} &= \frac{p_{rel}^2}{2 \mu k_B T} 												&\quad 		\tilde{E}  &= \frac{P_c^2}{2 M k_B T} \\
	p 				 &= \sqrt{2 \mu k_B T \tilde{\epsilon}}    										&\quad  	P_c 	   &= \sqrt{2 M k_B T \tilde{E}} \\
	dp p^2 			 &= \frac{\sqrt{\tilde{\epsilon}}}{2}(2 \mu k_B T)^{3/2} d \tilde{\epsilon} 	&\quad		dP_c \, P_c^2 &= \frac{\sqrt{\tilde{E}}}{2}(2 M k_B T)^{3/2} d \tilde{E}
\end{align*}

Plugging these expressions into Eq.\ref{eq:mom_dist_st_3} and rearranging

\begin{multline}
	\tilde{f}_{\vec{r}}(\vec{p}) = A^2 \frac{e^{-\tilde{\epsilon}}}{(2 \pi \mu k_B T)^{3/2}} \int_0^1 dX \frac{2}{\sqrt{\pi}} \int_0^\infty d \tilde{E} e^{-\tilde{E}}\sqrt{\tilde{E}} \\
	\times \heaviside{\eta(\vec{r}) - \frac{\tilde{E}}{2} - \frac{\tilde{\epsilon}}{2} - X \sqrt{\tilde{E}\tilde{\epsilon}}}
\end{multline}

would like to turn this distribution into a relative energy distrbution. Collisions are isotropic so we can use the relation

\begin{equation}
\begin{split}
	\int dp p^2 \int d \Omega_p \tilde{f}_{\vec{r}}(\vec{p}) = \int d \tilde{\epsilon} \hat{f}_{\vec{r}}(\tilde{\epsilon})) = 1 \\
	\Rightarrow 4 \pi p^2 \tilde{f}_{\vec{r}}(\vec{p}) dp = \hat{f}_{\vec{r}}(\tilde{\epsilon}) d \tilde{\epsilon}
\end{split}
\end{equation}

using $dp p^2$ given above we then write 

\begin{equation} \label{eq:mom_dist_st_4}
	\hat{f}_{\vec{r}}(\tilde{\epsilon}) = A^2 \sqrt{\tilde{\epsilon}} \, e^{-\tilde{\epsilon}} \frac{2}{\sqrt{\pi}} \int_0^1 dX \frac{2}{\sqrt{\pi}} \int_0^\infty d \tilde{E} e^{-\tilde{E}}\sqrt{\tilde{E}} \; \heaviside{\eta(\vec{r}) - \frac{\tilde{E}}{2} - \frac{\tilde{\epsilon}}{2} - X \sqrt{\tilde{E}\tilde{\epsilon}}}
\end{equation}

We can now choose the normalization constant $A^2$ using

\begin{equation*}
	\int_0^{2\eta(\vec{r})} d \tilde{\epsilon} \hat{f}_{\vec{r}}(\tilde{\epsilon}) = 1
\end{equation*}

where we have used an energy cutoff of $2 \eta(\vec{r})$ since either particle may have an energy in the range $[0 \rightarrow \eta(\vec{r})]$. With the normalization, the complete expression for $\hat{f}_{\vec{r}}(\tilde{\epsilon})$ is then

\begin{equation}
	\hat{f}_{\vec{r}}(\tilde{\epsilon}) = \frac{2}{\sqrt{\pi}} \sqrt{\tilde{\epsilon}} \, e^{-\tilde{\epsilon}} \hat{\mathcal{G}}(\eta_{\vec{r}}, \tilde{\epsilon})
\end{equation}

where all the effects of the truncation have been moved to $\hat{\mathcal{G}}$, given by 

\begin{equation*}
	\hat{\mathcal{G}}(\eta_{\vec{r}}, \tilde{\epsilon}) = \frac{\displaystyle
	\int_0^1 dX \frac{2}{\sqrt{\pi}} \int_0^\infty d \tilde{E} e^{-\tilde{E}}\sqrt{\tilde{E}} \; \heaviside{\eta(\vec{r}) - \frac{\tilde{E}}{2} - \frac{\tilde{\epsilon}}{2} - X \sqrt{\tilde{E}\tilde{\epsilon}}}}
	{\displaystyle \int_0^{2\eta(\vec{r})} d \tilde{\epsilon} \frac{2}{\sqrt{\pi}} \sqrt{\tilde{\epsilon}} \, e^{-\tilde{\epsilon}} \int_0^1 dX \frac{2}{\sqrt{\pi}} \int_0^\infty d \tilde{E} e^{-\tilde{E}}\sqrt{\tilde{E}} \; \heaviside{\eta(\vec{r}) - \frac{\tilde{E}}{2} - \frac{\tilde{\epsilon}}{2} - X \sqrt{\tilde{E}\tilde{\epsilon}}}}
\end{equation*}

we can check the limiting behavior of this equation since we expect when $$\lim_{\eta \rightarrow \infty} \hat{\mathcal{G}}(\eta_{\vec{r}}, \tilde{\epsilon}) = 1$$. Indeed, remembering that $$ \int_0^{\infty} dx \sqrt{x} e^{-x} = \frac{\sqrt{\pi}}{2}$$ then this requirement is fulfilled. 