\chapter{Progress towards studies of quantum magnetism}
\label{ch:chap6}

A straightforward extension of the work presneted in this thesis would be to control interparticle spacing via an optical lattice. For these and additional experiments using quantum degenerate fermionic strontium we purchased and installed an optical lattice system. Our lattice is implemented using a Coherent Verdi V-18 which is shapped and propagated to our science chamber in free space. \hl{Fig} shows the optical path for each arm of our cubic lattice. 

Unfortunately, complications due to heating when loading the lattice has limited our success in this optical trap. I want to go over what we have been able to do so far with the lattice.

How did we characterize?
	Kaptiza-dirac extension
	
What convinced us we were having problems?

What are some ideas we could do in the lattice?
	Zeno
	faster cooling via stimulated raman potentailly? (can I model this somehow?)
	repulzively bound molecules?
	use interaction control in lattice with the zeno thing
	
	

\section{Optical lattice: installation and characterization}
\label{sec:lattice}

"Lorem ipsum dolor sit amet, consectetur adipiscing elit, sed do eiusmod tempor incididunt ut labore et dolore magna aliqua. Ut enim ad minim veniam, quis nostrud exercitation ullamco laboris nisi ut aliquip ex ea commodo consequat. Duis aute irure dolor in reprehenderit in voluptate velit esse cillum dolore eu fugiat nulla pariatur. Excepteur sint occaecat cupidatat non proident, sunt in culpa qui officia deserunt mollit anim id est laborum."

\subsection{Background}
\label{ssec:lattice_background}

"Lorem ipsum dolor sit amet, consectetur adipiscing elit, sed do eiusmod tempor incididunt ut labore et dolore magna aliqua. Ut enim ad minim veniam, quis nostrud exercitation ullamco laboris nisi ut aliquip ex ea commodo consequat. Duis aute irure dolor in reprehenderit in voluptate velit esse cillum dolore eu fugiat nulla pariatur. Excepteur sint occaecat cupidatat non proident, sunt in culpa qui officia deserunt mollit anim id est laborum."

\subsection{Setup}
\label{ssec:lattice_setup}

"Lorem ipsum dolor sit amet, consectetur adipiscing elit, sed do eiusmod tempor incididunt ut labore et dolore magna aliqua. Ut enim ad minim veniam, quis nostrud exercitation ullamco laboris nisi ut aliquip ex ea commodo consequat. Duis aute irure dolor in reprehenderit in voluptate velit esse cillum dolore eu fugiat nulla pariatur. Excepteur sint occaecat cupidatat non proident, sunt in culpa qui officia deserunt mollit anim id est laborum."

\subsection{Measurement and results}
\label{ssec:lattice_tests}

"Lorem ipsum dolor sit amet, consectetur adipiscing elit, sed do eiusmod tempor incididunt ut labore et dolore magna aliqua. Ut enim ad minim veniam, quis nostrud exercitation ullamco laboris nisi ut aliquip ex ea commodo consequat. Duis aute irure dolor in reprehenderit in voluptate velit esse cillum dolore eu fugiat nulla pariatur. Excepteur sint occaecat cupidatat non proident, sunt in culpa qui officia deserunt mollit anim id est laborum."

\section{Spin manipulation of $^{87}$Sr}
\label{sec:spin_pol}

Here is where I need to introduce and characterize the \hl{LCR}

Averaging images together (how to use this code specifcally)

\section{Search for narrowline PA molecules using various spin mixtures}
\label{sec:87PAS}

"Lorem ipsum dolor sit amet, consectetur adipiscing elit, sed do eiusmod tempor incididunt ut labore et dolore magna aliqua. Ut enim ad minim veniam, quis nostrud exercitation ullamco laboris nisi ut aliquip ex ea commodo consequat. Duis aute irure dolor in reprehenderit in voluptate velit esse cillum dolore eu fugiat nulla pariatur. Excepteur sint occaecat cupidatat non proident, sunt in culpa qui officia deserunt mollit anim id est laborum."