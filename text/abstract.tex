\begin{abstract}

This dissertation describes experiments utilizing two-photon photoassociation to the least-bound vibrational level of the X$^1\Sigma_g^+$ electronic ground state of the $^{86}$Sr$_2$ dimer.
We measure the binding energy of this state to be $E_b=83.00(7)(20)$\,kHz, which represents the first observation of this naturally occurring halo molecule.
Using the precision of the halo state binding energy combined with the universal theory for a very weakly bound state on a potential that asymptotes to a van der Waals form, we determine $s$-wave scattering lengths for all strontium isotopes which are consistent with, but substantially more accurate than the previously determined values found from spectroscopic determination of long-range coefficients using other strontium isotopes.

With a radial expectation value of $\langle r \rangle \approx 21$\,nm the halo state extends well into the classically forbidden region which results in a large sensitivity of the dimer binding energy to light near-resonant with a bound-bound transition on the metastable $^1S_0-{^3P_1}$ potential.
This suggests that $^{86}$Sr may be a promising candidate for manipulating atomic interactions via optical coupling of the molecular bound states and probing naturally occurring Efimov states.
Furthermore, we observe novel multi-photon spectral loss features due to strong coupling and small detuning of the photoassociation lasers.
Using numerical simulations of a simple three-level model undergoing a simultaneous two-frequency drive, we show that solutions of the time-dependent Schr\"{o}dinger equation accurately reproduce these additional loss features and utilizing a Floquet analysis of this model, develop analytic formulas for predicting the AC Stark shift of the halo molecule.

Additionally, we report on the current state and recent modifications of our ultracold neutral strontium apparatus with a particular emphasis on the installation of a $532$\,nm optical lattice.
Characterization of this laser system and applicable use cases are discussed for further investigation of the $86$ halo molecule and additional photoassociation experiments.

\end{abstract}


