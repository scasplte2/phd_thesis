\chapter{Binding energy of the $^{86}$Sr$_2$ halo molecule}
\label{ch:chap4}

\section{Probing the ground state potential}
\label{sec:lowE_intro}

Strontium is a nice atom to work with because of the various properties of all of it's isotopes but access to such a variety of properties also comes with it share of complications. The most abundant isotope, 88, has a nearly vanishing scattering length but served as the workhorse for many of our previous stuides and those of other labs. The known scattering properties of strontium are mass scaled from 88 \hl{is this somehow not as good for 86? Also, where are the most up to date scattering lengths for Sr from? - The '10 Fourier paper} but by probing the 86 ground state potential directly we can obtain a more accurate measurement of the 86-86 scattering length.

Additionally, by utilizing the narrow intercombination potential we are able to detune many linewidths from the intermediate state, thereby 

Two-photon photoassociation 

umm, I want to write something about the narrow line letting us probe this but 

\section{Experimental setup}
\label{sec:lowE_setup}

Regarding the general process 

Using the \intPot{\gs}{\ex} interatomic potential, we perform a raman process using the $\nu=-2$ bound state which has a binding energy of E$_b=-44 \text{MHz}$ \hl{cite improved binding en}. Sample pre 

We used strontium 86 in a thermal gas at temperatures between 30 and 1000 nK. Typical peak densitieis were around $\peakDens{1}{12}$. 

Raman process using the second bound state of the \intPot{\gs}{\ex} interatomic potential

What atom do we use
what is the sample conditions
what trap do we use
what are the dimensions of the trap
what are the trap freq
how do we determine 

\subsection{Consideration of the trap depth}
\label{sec:trunc_trap}

Show plots of the trap and how we determined what the trap depth was

Our previous discussion of the rate loss constant assumed we could describe the spatial distribution of the atomic density profile analytically. This is a valid supposition given two key assumptions, 1) the sample termperature remains constant during the PA exposure and 2) the trap is of sufficient depth that we can reasonably approximate it as a harmonic trap. \hl{cite Mi's trap paper}.

Analysis of the trapping conditions after acquisition of the data revealed that this second assumption was not maintained during our experiment. In some figure we can see that we only have an eta of 2. This is troublesome as it means we must numerically consider the density distribution over space when solvinf for the rate loss constant K. 

In addition to the modified spatial distribution, we must also consider the effect of the trap depth on the energy profile of the trapped gas. In a typical high-eta trap, a typical Boltzmann profile is sufficient to describe the velocity distribution of the atoms and when we consider the distribution of relative energies that is important for PAS expeirments, we recover a simple bolztmann weighting for the distribution of energy probabilities. This is shown in \hl{sopme app}.

However, the case of a low-eta trap we must define a local cutoff energy at each point in space within the trap as atoms that have an energy higher than the local eta value are assumed to be lost from the trap. Derivation of this truncated relative energy probabillity distribution is given in \hl{some app} and results in a stronger weighting of the coldest atoms near the bottom of the trap.



\section{Theoretical description}
\label{sec:lowE_theory}

This section develops the more groddy form of the BJ equation. Include the 

In Ch. \hl{somewhere} we discussed the usual situation for observing loss due to photoassocition. This experiment was similar to the 88 autler townes experiment. 

\section{Spectral fitting and determination of the binding energy}
\label{sec:lowE_Eb2}

"Lorem ipsum dolor sit amet, consectetur adipiscing elit, sed do eiusmod tempor incididunt ut labore et dolore magna aliqua. Ut enim ad minim veniam, quis nostrud exercitation ullamco laboris nisi ut aliquip ex ea commodo consequat. Duis aute irure dolor in reprehenderit in voluptate velit esse cillum dolore eu fugiat nulla pariatur. Excepteur sint occaecat cupidatat non proident, sunt in culpa qui officia deserunt mollit anim id est laborum."

\section{Discussion of binding energy}
\label{sec:lowE_alt}

"Lorem ipsum dolor sit amet, consectetur adipiscing elit, sed do eiusmod tempor incididunt ut labore et dolore magna aliqua. Ut enim ad minim veniam, quis nostrud exercitation ullamco laboris nisi ut aliquip ex ea commodo consequat. Duis aute irure dolor in reprehenderit in voluptate velit esse cillum dolore eu fugiat nulla pariatur. Excepteur sint occaecat cupidatat non proident, sunt in culpa qui officia deserunt mollit anim id est laborum."

\section{Calculating the bound-bound Frank-Condon factor}
\label{sec:lowE_coupling}




As described in previous chapters, two-photon PAS can be used to directly populate low-lying molecular levels. Applying this technique to strontium 86 we can explore a similar regime



conclusion of chapter 4 is that we measured the binding energy more accurately which can be directly related to a more precious value of the scattering length for 86. Also there is a straightforward experiment available to use to attempt to measure the efimov paramter for strontium.

how did we determine the binding energy? how did we measure spectra?



From the formula for the line shape we can see that it depends on the spatial distribution of the atoms. The standard approximation made when measuring these types of systems is to ensure loss does not cause heating of the atoms during photoassociation. Heating results in re-equilibration of the atomic density distribition, which in turn effects the rate of loss creation. Without independent controls to keep the system in thermal (and therefore spatial density) equlibrium.

What are the things the rate equation deals with?

We need the desnity distribution.

In a harmonic trap, there if a simp[le anayltic form to the density distribution of a thermal gas. From Mi's work (and others) we know that this is only an approximation that is valid when eta is approx greater than 4. When greater than 4 we can apply the high-eta approx and the trap frequencies along a particular direction reduce to <eq>.

However, the trap we did this experiment in were at eta's of 1 or less so we don't have an analytic solution to the spatial distribution. Since this could be a problem we need to know what the trap looks like.

We measure trap oscillation freq. at several different powers and model the trap using the utility outlined somewhere else.

From the numeric model, we can define a spatially depedent eta which is determined by the local trap depth which is simply the difference between the local potential energy and the global depth. This is illustrated in fig something.

The spatial information is not only important for the density determination, but also for the range of available thermal energies. Consider two atoms near the local bottom of the trap. By definition, in equilibrium, a single atom may only have up to the trap depths worth of energy since any additional energy would result in its expulsion from the trap. In this case, in a relative momentum frame, the allowed collision energies range from zero to two times the trap depth. Similarly, as we move towards the edge of the trap the range of accessible collision energies shrinks. This additional weighting factor may be viewed as having a local truncasted Boltzmann ditribution at every point in space. 

Normally the BZ dist goes to infinity but here we have a cutoff at 2 trap depeth. The most naive approx would be to simply consider the BZ and harshly truncate at 2 trap depth. We tried this

We know this is unphysical since we should expect that the probability of observing a certain momenta at a certain point in space, should smoothly tend zero towards as we approach the edge of the trap. To see what this looks like we (and determine how important the effect is) we rederive the relative momentum distribution.

<Some stuff about center of mass and relative>

What were all the cases and conclusions of having done this? Remember to consider what the different cases are. If the total relative energy can be X then how does that get split up? Use the plots to show this limiting behavior. Like if particle 1 has all the energy then there is only one possible value for particle 2 (and vice versa).


DERIVATION for truncated trap below

Need to lookup references for this molecular chaos assumption. What about egodicity? How to discuss that we may not be completely ergodic?

What does my potential look like? Can I make it a piecewise function? How should I introduce this part?

Where does the f equation come from? I believe this is just the normalized boltzmann factor for probability to occupy a particlar state.

\noindent
We can truncate this single particle distribution by 
\begin{equation}
\label{eq:trun_single_particle_prob}
		 f_{ \vec{r} }( \vec{p} ) = A \left(\frac{1}{2 \pi k_B T}\right)^{3/2} e^{\left(\frac{-p^2}{2 m k_B T}\right)} \Theta \left( \epsilon_{max} - U( \vec{r} ) - \frac{p^2}{2 m} \right)
\end{equation}

\noindent
where A is a normalization constant which ensures $\int_0^\infty f_{ \vec{r} }( \vec{p} )\,d \vec{p} = 1 $ and $\Theta(x)$ is the Heaviside function defined by

\begin{equation}
\label{eq:heaviside}
	\Theta(x)=
	\begin{cases}
		1 &\text{if } x \geq 0, \\
		0 &\text{if } x < 0
	\end{cases}
\end{equation}

We got a certain answer with the way shown in the paper.

We can also use a completely different method that ignores all the consdierations of the last few sections. As was done in the calcium paper, we could simply fit the blue edge of the feature using a model function which can capture the high level features of the lineshape. Get the same answer. SHOW PLOTS TO THIS AFFECT AND COMPARE



Maybe go a little into the isolated resonance model (or at least recall), then tie into how we can measure the susceptibility across several different detunings which can give us the coupling to intermediate level. The first order analysis of this data suggest a bound-bound rabi freuqnecy of \hl{BLAH}. 

Point out the curling up at the end and say how the simple isolated resonacne model cannot predict. A full coupled channel calculation probably could but in the spirit of the Bohn and Julienne semi-classical approach, we set out to derive an approximate analytic expression to determine the binding energies. THis is presented in the next chapter.








Lastly, we note that in the context of photoassociation, the center-of-mass component of Eq.\ref{eq:two_particle_prob_inf_atomFrame} is not typically considered as typical PAS experiments are performed utilizing broad dipole allowed transitions which have linewidhts much greater than the doppler width thus only the relative momentum between particles is important for determining the loss rate coefficeint K discussed in \hl(somewhere). 



The case of PA using narrow intercombination line transitions found in alkaline-earth-metal atoms 


In general K is considered as a boltzmann average over a single loss rate constant
This can be seen in \cite{Ciuryo2004} Eq. 1 where the loss rate constant is given by

\begin{equation}
\begin{split}
\label{eq:ciuryo04_eq1}
		 K(\Delta,T) &= \left\langle\mathcal{K}(\Delta,\vec{P}_c,\vec{p}_r)\right\rangle \\
		 &= \int d^3\vec{P}_c \; f_M(\vec{P}_c) \int d^3\vec{p}_r \; f_{\mu}(\vec{p}_r) \; \mathcal{K}(\Delta,\vec{P}_c,\vec{p}_r)
\end{split}
\end{equation}



To this end we can integrate out the center of mass component to obtain the distribution most typically relevant to photoassociation.

By the time I've gotten to this I have already introduced K and that is not what I wanted to do. 

conclusion
here is the modified version of K we need for a trap that has a truncated energy disttribution

to get there
normal version of K is \hl{given in ch3}
this K can be given in terms of f? 
this ver4sion of f is given in the appendix
	why do I integrate out the com component?




typical PAS experiments utilize dipole allowed transitions which have linewidths many times larger than the 



We now perform a change of variables using Eq. and Eq.\ref{eq:two_particle_prob} can be rewritten as 




Need to make a connection between dipole matrix element, wigner threshold, and infinite squarer step potential. This infiinite square step can be viewed as a dilute ideal gas. 

To prove this assumption I want to show that using the square step I can get the same equations like in Eq. 1 of the 99 paper. Then once we know the infinite energy behavior (valid for only a particular portion of energy due to s-wave constraint) then we can ask what happens if f(p) is truncated. 

In the s-wave limit I need to write K as a function of f(p) (should do this in the appendix proof and reference in body). Given the form of the loss rate constant K, our problem reduces to determining the form of f(p) when eta is finite.

Ok, so need to reference \cite{Ciuryo2004} to motivate usage of center of mass. Then use \cite{Nicholson2015a} Eq. 43 to reference the particular form 


what is the throughline I want to make? Develop K$_{in}$ $\rightarrow$ recast in terms of P distribution $\rightarrow$ show how we can replace the normal dist with a truncated dist $\rightarrow$ explore the effects of that truncation 