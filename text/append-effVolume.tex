\chapter{Concise derivation of effective volumes}
\label{app:effective_volumes}

The following derivation is meant to serve as a quick reference for finding the analytic form of the effective volumes for ultracold gases held in a optical dipole trap.
This section follows the arguments presented in Mi Yan's work on modeling of collisions in an ODT \cite{Yan2011}.
This work numerically evaluates the general case of power-law potentials and the corresponding density distribution at arbitrary temperatures less than the trap depth.

If instead one restricts to the experimentally reasonable conditions of high-$\eta$ (recall $\eta$ is the ratio of trap depth to sample temperature, $\eta=\epsilon_t/k_B T$) and harmonic trapping potentials, then a useful analytic expression can be found for the effective volumes of the gas.

Starting with the definition of effective volumes
\begin{equation}
	V_q = \frac{1}{n_\text{peak}^q} \int d^3\vec{r} [n(\vec{r})]^q
\end{equation}
Defining $\eta = \epsilon_t / k_B T$, and integrating over all space we can formally write the number distribution as
\begin{align}
	n(\vec{r}) &= n_\text{peak} A \,\exp{\frac{-U(\vec{r})}{k_B T}} \left[ \text{erf}\left( \sqrt{\eta - \frac{U(\vec{r})}{k_B T}} \right) - 2 \sqrt{\frac{1}{\pi}\left( \eta - \frac{U(\vec{r})}{k_B T} \right)}\,\exp{-\eta +  \frac{U(\vec{r})}{k_B T}} \right] \\
	n_\text{peak} &= n_0 \left[ \text{erf}\left( \sqrt{\eta} \right) - 2 \sqrt{\frac{\eta}{\pi}}\,\exp{-\eta} \right]
\end{align}
where A is a normalization constant defined by
\begin{align*}
	A &= \frac{n_0}{n_\text{peak}} \\
   	  &= \left[ \text{erf}\left( \sqrt{\eta} \right) - 2 \sqrt{\frac{\eta}{\pi}}\,\exp{-\eta} \right]^{-1}
\end{align*}
Plugging these equations into the expression for the effective volume we find
\begin{equation}
	V_q = \frac{1}{[\mathcal{P}(\eta,3/2)]^q} \int d^3\vec{r}\, \exp{-q \frac{U(\vec{r})}{k_B T}} \left[ \mathcal{P}(\eta - \frac{U(\vec{r})}{k_B T}, 3/2) \right]^q
\end{equation}
where $\mathcal{P}(\eta, a)$ is the incomplete Gamma function.
It is easily verifiable in Mathematica that for $\eta \geq 4$ then $\mathcal{P}(\eta,a)\rightarrow1$.
Thus in a deeply trapped regime, we can approximate the effective volume as
\begin{equation}
	V_q = \int d^3\vec{r} \,\exp{-q\frac{U(\vec{r})}{k_B T}}
\end{equation}

Considering an anisotropic harmonic potential $U(x,y,z) = \frac{1}{2}m(\omega_x^2 x^2 + \omega_y^2 y^2 + \omega_z^2 z^2)$, then we can find the characteristic length as
\begin{equation}
	R_i^2 = \frac{2 \epsilon_t}{m \omega_i^2}
\end{equation}
where $\epsilon_t$ is the trap depth determined by the lowest saddle point.
Thus the effective volume may be written
\begin{equation}
	V_q = \int_{-R_x}^{R_x} dx \int_{-R_y}^{R_y} dy \int_{-R_z}^{R_z} dz  \,\exp{-q \frac{m(\omega_x^2 x^2 + \omega_y^2 y^2 + \omega_z^2 z^2)}{2 k_B T}}
\end{equation}
Making a change of variables to $\tilde{r_i} = r_i/R_i$ for each axis we may rewrite $V_q$ as
\begin{align}
	V_q &= R_x R_y R_z \int_{-1}^{1} d\tilde{x} \int_{-1}^{1} d\tilde{y} \int_{-1}^{1} d\tilde{z}  \,\exp{-q \eta(\tilde{x}^2 + \tilde{y}^2 + \tilde{z}^2)} \\
	&= R_x R_y R_z \left( \frac{\pi}{q \eta} \right)^{3/2} [\text{erf}(\sqrt{q \eta})]^3
\end{align}
where recall that $\eta = \epsilon_t/k_B T$.
Once more, it is readily verified for $\eta \gg 1$ then $\text{erf}(\sqrt{q \eta})\rightarrow1$ and therefore
\begin{equation}
	V_q = \left[ \frac{2 \pi k_B T}{q m \bar{\omega}} \right]^{3/2}
\end{equation}
where $\bar{\omega} = (\omega_x \omega_y \omega_z)^{1/3}$ is geometric mean of the trap frequencies. 